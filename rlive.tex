\documentclass{article}
\usepackage{amsmath}
\usepackage{fullpage}
\usepackage{hyperref}
\usepackage{amsmath}
\usepackage{amssymb}
\usepackage{xspace}
\usepackage{xcolor}

\newcommand{\red}[1]{\textcolor{red}{#1}}
\newcommand{\blue}[1]{\textcolor{blue}{#1}}

\usepackage[ruled,linesnumbered,commentsnumbered,noend]{algorithm2e}
  \SetKwInput{KwData}{Var}
  \SetKwInput{KwInit}{Init}
  \SetKwComment{comment}{$\LHD$}{}
  \SetKwInOut{Input}{Input}
  \SetKwInOut{Output}{Output}
  \DontPrintSemicolon

\newcommand{\rlive}{\textsf{rlive}\xspace}
\newcommand{\icccia}{\textsf{ic3ia}}
\newcommand{\iccciafixed}{\textsf{ic3ia-fixed}}
\newcommand{\iccciaabs}{\textsf{ic3ia-abs}}
\newcommand{\iccc}{\textsf{ic3}}
\newcommand{\concr}{\textsf{concretize}}
\newcommand{\mca}{\ensuremath{\textsf{model-check}_\alpha}}
\newcommand{\mc}{\textsf{model-check}}
\newcommand{\abs}[1]{\ensuremath{\widehat{#1}}}
\newcommand{\absproc}{\textsf{abstract}}
\newcommand{\preds}{\ensuremath{\mathbb{P}}}
\newcommand{\N}{\ensuremath{\mathbb{N}}}

\newcommand{\mcconc}{\textsf{model-check-concrete}}
\newcommand{\mcabs}{\textsf{model-check-abstract}}
\newcommand{\getmcinv}{\textsf{get-mc-inv}}
\newcommand{\getmccex}{\textsf{get-mc-cex}}
\newcommand{\getabsmccex}{\textsf{get-abs-mc-cex}}
\newcommand{\getrliveshoals}{\textsf{get-rlive-shoals}}
\newcommand{\getrlivecex}{\textsf{get-rlive-cex}}
\newcommand{\feasible}{\textsf{feasible}}
\newcommand{\feasibleloop}{\textsf{feasible-loop}}
\newcommand{\refine}{\textsf{refine}}
\newcommand{\synthrank}{\textsf{synth-rank-function}}

\begin{document}

\title{Infinite-state rlive}
\author{Christopher Johannsen}
\maketitle

\section{Preliminaries}

We focus on model checking of infinite state systems. A system is defined as $S = \langle X,I,T
\rangle$ where $X$ is a set of variables, $I$ is a one-state formula over $X$ defining the initial
condition and $T$ is a two-state formula over $X,X'$ defining the transition relation where $X' =
\{x'\}_{x \in X}$ is a set of next-state variables. A trace $\pi = s_0,s_1,\dots$ of the system $S$
is a sequence of states such that $s_0 \models I$ and $s_i \land T \models s_{i+1}'$. 

\subsection{Invariant Model checking}

The invariant model checking problem asks whether a property $P$ holds in all reachable states of a
system $S = \langle I,T \rangle$. A procedure $\mc(I,T,P)$ will return a finite trace $\pi =
s_0,s_1,\dots,s_n$ of $S$ where $s_n \models \neg P$, or an inductive invariant $inv$ showing that
$inv \models P$, $I \models inv$, and $inv \land T \models inv'$.

\subsection{Abstract Invariant Model Checking}

Abstract model checking takes a concrete system and a set of predicates \preds{} then returns either
the result \emph{unsafe} and an abstract trace $\abs{\pi} = \abs{s_0},\abs{s_1},\dots$ over \preds{}
that violates the property or the result \emph{safe} and an inductive invariant $\abs{inv}$ over
\preds.

We assume access to a procedure $\mcabs(I,T,P,\preds)$ that returns an abstract trace $\abs{\pi}$ in
case of violation of $P$ or an inductive invariant $inv$ over \preds{} otherwise. 

\subsection{Liveness Checking}

We focus on properties of the form $FGq$, i.e., liveness properties. The problem under consideration
is whether some system $S$ satisfies a property $P = FGq$, in other words, we compute $S \models
FGq$. When $S$ is finite, $S \nvDash FGq$ if and only if there exists a loop $\alpha \cdot
\beta^\omega$ in $S$ where some state $b \in \beta$ is such that $b \models \neg q$. If $S$ is
infinite, then the existence of a loop $\alpha \cdot \beta^\omega$ proves that $S \nvDash FGq$, but
there may be other non-loop counterexamples.

\subsection{\rlive}

One algorithm for checking liveness properties is \rlive. As opposed to other algorithms
(Liveness-to-safety, FAIR, k-liveness, k-FAIR), \rlive performs a depth-first search for a loop that
violates $FGq$, i.e., one that satisfies $GF \neg q$. 

The algorithm incrementally searches for a path that satisfies $\neg q$ infinitely often via a
series of invariant model checking queries. If the algorithm determines that $\neg q$ is unreachable
from the current state in the search, \rlive adds the newly-found inductive invariant to a set of
states known as \emph{shoals}, otherwise it reaches another state $s$ that satisfies $\neg q$. If
$s$ has already been seen, then a loop satisfying $GF\neg q$ as been found, so \rlive terminates
with \emph{unsafe}. Otherwise \rlive adds this state to its stack and continues the search. The
algorithm terminates with \emph{safe} once it shows that the initial set of states is in the
\emph{shoals}. We present a version of the algorithm in Figure~\ref{alg:rlive}.

\begin{algorithm}[t]
\SetKwFunction{KwFn}{Procedure \textsf{rlive}}
\KwFn{$I$, $T$, $FGq$, $C_{init}$}
\Begin{
    $C := C_{init}$ \tcp{initial shoals}
    $B := $ empty stack of states \;
    \While{$\mcconc(I, T \land (\neg C \land \neg C'), \neg q \land T^{-1}(\neg C))$ is unsafe}{
        $s := $ final state of \getmccex() \;
        $B.push(s)$ \;
        \While{$B$ is not empty}{
            $s := B.top()$ \;
            \If{$s \in B$}{
                \Return{(Unsafe, $C$)}
            } 
            \If{$\mcconc(T(s), T \land (\neg C \land \neg C'), \neg q \land T^{-1}(\neg C))$ is unsafe} {
                $t := $ final state of \getmccex() \;
                $B.push(t)$ \;
            } \Else {
                $inv := \getmcinv()$ \;
                $C := C \lor inv$ \; 
                $B.pop()$ \;
            }
        }
    }
    \Return{Safe}
}
\caption{\rlive algorithm}
\label{alg:rlive}
\end{algorithm}

\section{Algorithms}

Here we give an overview of the proposed algorithms.

\begin{table}[h]
\centering
\begin{tabular}{|c|c|c|c|c|}
\hline
\textbf{Alg.} & \textbf{Model Abstraction} & \textbf{Refinement} & \textbf{Predicates} \\ 
\hline \hline
Basic         & N/A                        & N/A                 & N/A                 \\ \hline
EA-LR         & Explicit                   & Lazy                & Global              \\ \hline
IA-LR         & Implicit                   & Lazy                & Global              \\ \hline
IA-ER         & Implicit                   & Eager               & Global              \\ \hline
IA-ER*        & Implicit                   & Eager               & Local               \\ \hline
\end{tabular}
\end{table}

\begin{description}
    \item[Basic] runs \rlive as it is, except over the infinite state system. Functionally
    this means that a model checker will use SMT queries rather than SAT ones.

    \item[EA-LR] constructs an abstract system $\abs{S} = \langle \abs{I}, \abs{T} \rangle$
    explicitly over a fixed set of predicates $\preds$, using techniques like ALLSAT, then standard
    \rlive on the abstract system. If the query returns safe or a feasible counterexample, the
    corresponding result is returned. If the query returns an infeasible counterexample, the set of
    predicates is refined to eliminate this counterexample, the process is repeated with the newly
    refined set of predicates.

    \item[IA-LR] is the same as Algorithm 1 except that it uses implicit abstraction to
    avoid explicitly computing the abstract model.
    
    \item[IA-ER] performs model checking on the abstract state space using implicit
    abstraction, but checks that each resulting abstract counterexample is feasible. 
    
    \item[IA-ER*] uses a local set of predicates for each model checking query. In other words,
    the model checker returns both a counterexample and an updated set of predicates in the case of
    a violation, and the returned set of predicates is used as the initial set of predicates for the
    next model checking query. 
\end{description}

% Another candidate -- concrete model checking on abstract I,q and concrete T

% For algorithms 1-3, we must define functions \feasible{} and \refine{} for checking and blocking
% spurious counterexamples. In particular, \feasible{} will take as input $I$, $T$, \preds and
% $\abs{\pi}$ and output true/false while \refine{} will take as input $I$, $T$, \preds, and
% $\abs{\pi}$ and output a formula $R_T$ and new set of predicates \preds' such that 

The algorithms assume access to a procedure \mc($I$,$T$,$P$) that performs model checking for
property $P$ and returns a concrete counterexample in the case of violation, as well as
\mca($I$,$T$,$P$) that returns an abstract counterexample. \mca{} can also optionally take a set of
initial predicates as in \mca($I$,$T$,$P$,\preds). The procedure \absproc($\phi$, \preds) explicitly
constructs an abstract formula $\abs{\phi}$ using the predicates in \preds. \getrlivecex{} returns
the counterexample from \rlive{} separated into its prefix and lasso portions.

\begin{algorithm}[t]
\SetKwFunction{KwFn}{Procedure $\textsf{rlive}_{\textsf{EA-LR}}$}
\KwFn{$I$, $T$, $FGq$}
\Begin{
    $\preds := $ predicate symbols in $I$, $q$ \;
    $R_T,\abs{R_T} := true$ \tcp{formula for $T$ refinement}
    $\abs{I} := \absproc(I,\preds)$, $\abs{T} := \absproc(T,\preds)$, $\abs{q} := \absproc(q,\preds)$ \;
    $C := false$ \tcp{initialize shoals}
    \While{$\rlive(\abs{I},\abs{T} \land \abs{R_T},FG \abs{q}, C)$ is unsafe}{
        $C := \getrliveshoals()$ \; 
        $\abs{\pi_\alpha},\abs{\pi_\beta} := \getrlivecex()$ \;
        \If{$\feasible(T \land R_T, \preds, \abs{\pi_\alpha})$ and $\feasibleloop(T, \preds, \abs{\pi_\beta})$} {
            \Return{Unsafe}
        } 
        $R_T,\preds := \textsf{refine}(T,\preds,\abs{\pi})$ \;
        $\abs{I} := \absproc(I,\preds)$, $\abs{T} := \absproc(T,\preds)$, \; 
        $\abs{q} := \absproc(q,\preds)$, $\abs{R_T} := \absproc(R_T,\preds)$ \;
    }
    \Return{Safe}
}
\caption{Explicit abstraction, lazy refinement}
\label{alg:0a}
\end{algorithm}

\begin{algorithm}[t]
\SetKwFunction{KwFn}{Procedure $\textsf{rlive}_{\textsf{IA-LR}}$}
\KwFn{$I$, $T$, $FGq$}
\Begin{
    $\preds := $ predicate symbols in $I$, $q$ \;
    $R_T := true$ \tcp{formula for $T$ refinement}
    $C := false$ \tcp{initialize shoals}
    \While{$\textsf{rlive-rec}_{\textsf{IA-LR}}(I,T \land R_T,FGq,\preds,C)$ is unsafe}{
        $\abs{\pi_\alpha},\abs{\pi_\beta} := \getrlivecex()$ \;
        \If{$\feasible(T \land R_T, \preds, \abs{\pi_\alpha})$ and $\feasibleloop(T, \preds, \abs{\pi_\beta})$} {
            \Return{Unsafe}
        } 
        $R_T,\preds := \textsf{refine}(I,T \land R_T,\preds,\abs{\pi})$ \;
        $C := \getrliveshoals()$ \; 
    }
    \Return{Safe}
}
\;
\SetKwFunction{KwFn}{Procedure $\textsf{rlive-rec}_{\textsf{IA-LR}}$}
\KwFn{$I$, $T$, $FGq$, $C$}
\Begin{
    $\abs{B} := $ empty stack of abstract states \;
    \While{$\mcabs(I, T \land (\neg C \land \neg C'), \neg q \land T^{-1}(\neg C), \preds)$ is unsafe}{
        $\abs{s} := $ final state of $\getabsmccex()$ \;
        $\abs{B}.push(\abs{s})$ \;
        \While{$\abs{B}$ is not empty}{
            $\abs{s} := \abs{B}.top()$ \;
            \If{$\abs{s} \in \abs{B}$}{
                \Return{(Unsafe, $C$)}
            } 
            \If{$\mcabs(\concr(\abs{T}(\abs{s})), T \land (\neg C \land \neg C'), \neg q \land T^{-1}(\neg C), \preds)$ is unsafe} {
                $\abs{t} := $ final state of $\getabsmccex()$ \;
                $\abs{B}.push(\abs{t})$ \;
            } \Else {
                $\abs{B}.pop()$ \;
                $inv := \concr(\getmcinv())$ \;
                $C := C \lor inv$ \; 
            }
        }
    }
    \Return{Safe}
}
\caption{Implicit abstraction, lazy refinement}
\label{alg:0b}
\end{algorithm}

This is similar to the previous algorithm except we do \emph{eager refinement}, i.e., we check that
each path segment of the loop is feasible before continuing. If any path segment is infeasible, we
return to the CEGAR loop and refine. Otherwise, we continue until we find an abstract loop of the
form:
$$
    \abs{\pi_0} \cdot \abs{\pi_1} \cdots \abs{\pi_i} \cdots \abs{\pi_j}
$$
where the first state of $\abs{\pi_i}$ is the same as the last state of $\abs{\pi_j}$:
$\abs{\pi_i}[0] = \abs{\pi_j}[-1]$. The algorithm maintains the invariant that $\feasible(I, T,
\preds, \abs{\pi_k})$ is true for all path segments $k \in 0..j$, but does not guarantee that the
interfaces between them are feasible. In other words,
$$
\feasible(I,T,\preds,\abs{\pi_k}) \land \feasible(I,T,\preds,\abs{\pi_{k+1}}) \not\Rightarrow \feasible(I, T, \preds,
\abs{\pi_k} \cdot \abs{\pi_{k+1}}).
$$
However, the $\feasible$ check on line 21 will check this.

To solve this, we guarantee the feasibility of the full abstract trace found so far during the
search and check $\feasible(I, T, \preds, \abs{\pi_0} \cdots \abs{\pi_k})$ for each newly found path
segment $\abs{\pi_k}$.

\begin{algorithm}
\SetKwFunction{KwFn}{Procedure $\textsf{rlive}_{\textsf{IA-ER}}$}
\KwFn{$I$, $T$, $FGq$}
\Begin{
    $\preds := $ predicate symbols in $I$, $q$ \;
    \While{$\textsf{rlive-3-rec}(I,T,FGq,\preds)$ is infeasible}{
        $\abs{\Pi} :=$ infeasible counterexample from \textsf{rlive-3-rec} \;
        $I,T,\preds := \textsf{refine}(T,\preds,\abs{\Pi})$ \;
    }
    \Return{Result from \textsf{rlive-3-rec}}
}
\;
\SetKwFunction{KwFn}{Procedure $\textsf{rlive-rec}_{\textsf{IA-ER}}$}
\KwFn{$I,T,FGq,\preds$}
\Begin{
    $C := false$ \tcp{Shoals}
    $\abs{B} := $ empty stack of states \;
    $\abs{\Pi} := \emptyset$ \tcp{empty trace}
    \While{$\mcabs(I, T \land (\neg C \land \neg C'), \neg q \land T^{-1}(\neg C), \preds)$ is unsafe}{
        $\abs{\pi} := \getabsmccex()$ \;
        \If {$\neg \feasible(T \land (\neg C \land \neg C'), \preds, \abs{\pi})$} {
            \Return{Infeasible $\abs{\pi}$}
        }
        $\abs{s} :=$ final state of $\abs{\pi}$ \;
        $\abs{B}.push(\abs{s})$ \;
        \While{$\abs{B}$ is not empty}{
            $\abs{s} := \abs{B}.top()$ \;
            \If{$\abs{s} \in \abs{B}$}{
                $\abs{\Pi_\alpha} := \abs{\Pi}$ up to and including $\abs{s}$ \;
                $\abs{\Pi_\beta} := \abs{\Pi}$ including and after $\abs{s}$ \;
                \If{$\feasible(T \land (\neg C \land \neg C'),\preds,\abs{\Pi_\alpha})$ and $\feasibleloop(T \land (\neg C \land \neg C'),\preds,\abs{\Pi_\beta})$} {
                    \Return{Unsafe}
                } \Else{
                    \Return{Infeasible $\abs{\Pi}$}
                }
            } 
            \If{$\mcabs(T(s), T \land (\neg C \land \neg C'), \neg q \land T^{-1}(\neg C), \preds)$ is unsafe} {
                $\abs{\pi} := \getabsmccex()$ \;
                \If {$\neg \feasible(T \land (\neg C \land \neg C'), \preds, \abs{\Pi} \cdot \abs{\pi})$} {
                    \Return{Infeasible $\abs{\Pi}$}
                }
                $\abs{\Pi} := \abs{\Pi} \cdot \abs{\pi}$ \;
                $\abs{t} :=$ final state of $\abs{\pi}$ \;
                $\abs{B}.push(\abs{t})$ \;
            } \Else {
                $inv := $ invariant from \mcabs \;
                $C := C \lor inv$ \; 
                $\abs{B}.pop()$ \;
            }
        }
    }
    \Return{Safe}
}
\caption{Implicit abstraction, eager refinement}
\label{alg:3}
\end{algorithm}

\section{Abstract Counterexample Refinement}

There are four outcomes to consider given a system $S$. Let $\abs{\pi} = \alpha \cdot \beta^\omega$
be a candidate abstract counterexample.

\begin{enumerate}
    \item $\exists \pi : \pi = \concr(S,\abs{\pi})$, return such a $\pi$.
    \item $\not\exists \pi : \pi = \concr(S,\abs{\pi})$, refine predicates to block $\abs{\pi}$.
    Could use interpolants ala \icccia~\cite{cimatti2016infinite}.
    \item $\forall k : \exists \pi : \pi = \concr(S,\alpha \cdot \beta^k)$ and $\not\exists \pi :
    \pi = \concr(S,\alpha \cdot \beta^\omega)$, synthesize a ranking function and use this to block
    $\abs{\pi}$. We may consider using the techniques presented in \cite{cook2005abstraction}.
    \item There exists a non-lasso-shaped counterexample, use well-founded funnels to find these.
\end{enumerate}

The algorithms outlines previously only cover cases 1 and 2, but can be outfitted to handle the
other cases. For case 3, if we can synthesize such a ranking function, we must refine the system to
block the abstract loop in the next model checking calls. Let $\abs{\pi_0} \cdot \abs{\pi_1} \cdots
\abs{\pi_i}$ be an abstract loop made up of path segments such that there exists a ranking function
$R$ that includes every state in the loop. \red{Is there a naive approach to start from?}

\begin{algorithm}
\SetKwFunction{KwFn}{Procedure \feasible}
\KwFn{$T,\preds,\abs{\pi}$}
\Begin{
    $\abs{s_0},\dots,\abs{s_n} := \abs{\pi}$ \;
    \If{$\bigwedge_{0 \leq i < n} T(X^i, X^{i+1}) \land \bigwedge_{0 < i \leq n}
    \abs{s_i}(X^i_\preds)[X^i_\preds \mapsto \preds^i]$ is unsat}{
        \tcp{Compute sequence interpolant (Section 3.3~\cite{cimatti2016infinite})}
        \Return{New set of predicates}
    } \Else{
        \Return{Satisfiable assignment to if condition}
    }
}
\;
\SetKwFunction{KwFn}{Procedure \feasibleloop}
\KwFn{$T,\preds,\abs{\pi}$}
\Begin{
    $\abs{s_0},\dots,\abs{s_n} := \abs{\pi}$ \;
    \If{$\bigwedge_{0 \leq i < k} T(X^i, X^{i+1}) \land \bigwedge_{0 < i \leq k}
    \abs{s_i}(X^i_\preds)[X^i_\preds \mapsto \preds^i]$ is unsat}{
        \tcp{Compute sequence interpolant (Section 3.3~\cite{cimatti2016infinite})}
        \Return{New set of predicates}
    } \Else{
        $R := \synthrank(T,\abs{\pi})$ \tcp{Using technique on p11 of \cite{daniel2016infinite}}
        \tcp{\red{How to use $R$ to block $\abs{\pi}$ in $T$?}}
        \tcp{\red{For now, assume we find some formula $R_T$ such that $\abs{T} \land \abs{R_T}$ does not admit $\abs{\pi}$}}
        \Return{$R_T$}
    }
}
\caption{Feasibility checking}
\label{alg:feasible}
\end{algorithm}

\subsection{Example}

% TODO: Concrete example
% Consider a counter with property like I:= x=0 /\ n>0, T:= x'=x+1 /\ n'=n /\ (x==n)->q, P:= GFq 
% Then adding ranking function n-x to property like: GF(n-x>=0) -> G!q (?)

We consider an example system to illustrate how to use ranking functions to block spurious counterexamples in \rlive.
\begin{align*}
    V & = \{x,n\} \\
    I & := x = 0 \land n > 0 \\
    T & := x' = x + 1 \land n' = n \\
    P & := GF(x=n)
\end{align*}
In a naive version of the algorithm (Algorithm~\ref{alg:rlive}), the call to \mcconc{} will find  

A ranking function that can be used to prove $F(x=n)$ is $R(V) = n-x$.

\bibliographystyle{plain}
\bibliography{rlive}

\end{document}




% \begin{algorithm}
% \SetKwFunction{KwFn}{Procedure \textsf{rlive-4}}
% \KwFn{$I,T,FGq$}
% \Begin{
%     $C := false$ \tcp{Shoals}
%     $\preds := $ predicates in $I$, $q$ \;
%     \While{$\mcabs(I, T \land (\neg C \land \neg C'), \neg q \land T^{-1}(\neg C), \preds)$ is unsafe}{
%         $\abs{B} := $ empty stack of states \;
%         $\abs{\Pi} := \emptyset$ \tcp{empty trace}
%         $\abs{\pi} := \getabsmccex()$ \;
%         \If {$\neg \feasible(I, T \land (\neg C \land \neg C'), \preds, \abs{\pi})$} {
%             $I,T,\preds := \refine(I,T,FGq,\preds,\abs{\pi})$ \;
%             \textbf{continue} \;
%         }
%         $\abs{\Pi} := \abs{\pi}$ \;
%         $\abs{s} := \abs{\pi}[-1]$ \;
%         $\abs{B}.push(\abs{s})$ \;
%         \While{$\abs{B}$ is not empty}{
%             $\abs{s} := \abs{B}.pop()$ \;
%             \If{$\exists \abs{b} \in \abs{B} : \abs{s} \rightarrow \abs{b}$}{
%                 \If{$\feasible(I,T,\preds,\abs{\Pi})$} {
%                     \Return{Unsafe}
%                 } \Else{
%                     $I,T,\preds := \refine(I,T,FGq,\preds,\abs{\pi})$ \;
%                     \textbf{break} \tcp{jump to outer loop}
%                 }
%             } 
%             \If{$\mcabs(T(s), T \land (\neg C \land \neg C'), \neg q \land T^{-1}(\neg C), \preds)$ is unsafe} {
%                 $\abs{\pi} := \getabsmccex()$ \;
%                 \If {$\neg \feasible(I, T \land (\neg C \land \neg C'), \preds, \abs{\Pi} \cdot \abs{\pi})$} {
%                     $I,T,\preds := \refine(I,T,FGq,\preds,\abs{\pi})$ \;
%                     \textbf{break} \tcp{jump to outer loop}
%                 }
%                 $\abs{\Pi} := \abs{\Pi} \cdot \abs{\pi}$ \;
%                 $\abs{t} := \abs{\pi}[-1]$ \;
%                 $\abs{B}.push(\abs{s})$ \;
%                 $\abs{B}.push(\abs{t})$ \;
%             } \Else {
%                 $inv := $ invariant from \mcabs \;
%                 $C := C \lor inv$ \; 
%             }
%         }
%     }
%     \Return{Safe}
% }
% \caption{Eager refinement, local predicate refinement}
% \label{alg:4}
% \end{algorithm}